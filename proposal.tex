\documentclass{article}
\usepackage{geometry}
\usepackage{csquotes}
\usepackage{hyperref}
\geometry{legalpaper, portrait, margin=1in}
 
\title{Parallelizing Latent Dirichlet Allocation (LDA)}
\author{Virgile Audi (vaudi@g.harvard.edu)\\ 
		Nicolas Drizard (nicolasdrizard@g.harvard.edu)\\
		Charles Liu (cliu02@g.harvard.edu)}
\date{\today}
 
\begin{document}
 
\maketitle
 
\section{Introduction}
	\subsection{Background}
		Latent Dirichlet Allocation was a topic model \href{https://www.cs.princeton.edu/~blei/papers/BleiNgJordan2003.pdf}{proposed} in 2003 that allows sets of observations to be explained by topics automatically discovered by the LDA technique. As stated in the article:

		\begin{displayquote}
			The goal is to find short descriptions of the members of a collection that enable efficient
			processing of large collections while preserving the essential statistical relationships that are useful
			for basic tasks such as classification, novelty detection, summarization, and similarity and relevance
			judgments.
		\end{displayquote}

		The technique has widely been used in natural language processing.

	\subsection{Motivation}
		There is currently no public version in python that is parallelized (or in Cython). We think such functionality would be useful to the community, as LDA is a conceptually simple algorithm but provides impressive results.
 
\section{Objectives}
	\subsection{Functionality}
		The algorithm is well-defined, we are just trying to parallelize it. We'd like to provide a Cython implementation as well, and apply more novel parallelization techniques such as Hogwild's lock-free approach to gradient descent.

	\subsection{Performance}
		We will benchmark our algorithm against two existing libraries:
		\begin{itemize}
			\item Serial Python library: \url{https://pypi.python.org/pypi/lda}
			\item Parallel C++ library: \url{https://code.google.com/p/plda/}
		\end{itemize}
\section{Design Overview}
	\subsection{Technologies Used}
		As this is somewhat an embarassingly parallel problem, we want to look at two approaches:
		\begin{itemize}
			\item Python multiprocessing
			\item Cython + Python multithreading
		\end{itemize}

 	\subsection{Parallelism Details}
 		There are three areas where this can be parallelized:
 		\begin{itemize}
 			\item Gradient descent using the Hogwild approach
 			\item Variational inference
 			\item Gibbs sampler
 		\end{itemize}
 \section{Verification}
 	As there are serial implementations available we will compare with those over Yelp and Guttenberg data (and larger data sets we have yet to find).
 \section{Schedule}
 	\begin{itemize}
 		\item Functional, serial LDA implementation
 		\item Parallelized gradient descent using Hogwild
 		\item Parallelized update procedure (Gibbs sampling and variational inference)
 	\end{itemize}
 \section{Work Distribution}
 	Each member will write a serial LDA to understand the algorithm and compare performance. Since there are three parallelization parts, each member will focus on one.

\end{document}